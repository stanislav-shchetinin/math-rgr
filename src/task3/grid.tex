\begin{frame}\frametitle{Координатная сетка}
	Изучим, как под действием отображения изменяется координатная сетка:
	\begin{enumerate}
		\item Построим в виртуальном пространстве множество точек
		      \(v = C\)~--~горизонтальные прямые и \(u = C\)~--~вертикальные прямые
		\item Применим к этим точкам преобразование
		\item Изобразим получившиеся точки в физическом пространстве
	\end{enumerate}

	В приведенных на следующих слайдах графиках константа \(C \in \{-2, -1, -0.5, 0, 0.5, 1, 2\}\)
\end{frame}

\begin{frame}{Координатная сетка (горизонтальные прямые)}
	\begin{figure}
		\centering
		\includegraphics[width=0.9\textwidth]{figures/conformal_grid_horizontal.pdf}
		\caption{Координатная сетка (горизонтальные прямые)}\label{fig:conformal_grid_horizontal}
	\end{figure}
\end{frame}

\begin{frame}{Координатная сетка (вертикальные прямые)}
	\begin{figure}
		\centering
		\includegraphics[width=0.9\textwidth]{figures/conformal_grid_vertical.pdf}
		\caption{Координатная сетка (вертикальные прямые)}\label{fig:conformal_grid_vertical}
	\end{figure}
\end{frame}


\begin{frame}{Координатная сетка}
	\begin{figure}
		\centering
		\includegraphics[width=0.9\textwidth]{figures/conformal_grid_combined.pdf}
		\caption{Координатная сетка}\label{fig:conformal_grid}
	\end{figure}
\end{frame}
