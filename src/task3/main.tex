\begin{frame}\frametitle{Задание 3. Конформные отображения}
	\begin{equation*}
		w(z) = \frac{z-1}{z+1} = 1 - \frac{2}{z+1}
	\end{equation*}

	План выполнения работы:
	\begin{enumerate}
		\item Рассмотреть конформное отображение.
		      Определить особые точки отображения (при наличии) и указать их вид.

		\item Изобразить на комплексной плоскости отображение
		      области виртуального пространства в область физического пространства
		      с помощью заданного преобразования.

		\item Выделить действительную и мнимую части отображения
		      для построения искривленной координатной сетки в физическом пространстве.

		\item Взять обратное преобразование к заданному и проанализировать его

		\item Рассчитать профиль показателя преломления используя конформное отображение

	\end{enumerate}
\end{frame}

\subsection{Особые точки}
\begin{frame}\frametitle{Особые точки}
	Отображение имеет две особые точки \(z_1 = 1\) и \(z_2 = -1\).
	Определим их вид.
	Для этого найдем производную \(w'(z)\).

	\[
		w'(z) = \frac{2}{(z+1)^2}
		\qquad
		w(z_1) = w(1) = 0
		\quad
		w'(z_1) = w'(1) \neq 0
	\]
	Значит точка \(z_1 = 1\) является простым нулем.
	Определим вид точки \(z_2 = -1\).

	\[ \lim_{z \to -1} \frac{z-1}{z+1} = \infty \]

	Для функции \(g(z) = 1/w(z) = \frac{z+1}{z-1}\)
	точка \(z_2 = -1\) является простым нулем.
	Значит точка \(z_2 = -1\) является для функции \(w(z)\) полюсом первого порядка.

	Таким образом, отображение является конформным за исключением точки \(z = -1\)
\end{frame}

\subsection{Im(z) и Re(z)}
\begin{frame}{\(\Im w(z)\) и \(\Re w(z)\)}
	Для дальнейшего изучения отображения найдем \( \Im(w(z)) \) и \( \Re(w(z)) \).
	Пусть \( z = u + i v \). Тогда:

	\begin{align*}
		w(z) & = w(u + iv) = 1 - \frac{2}{(u + 1) + iv} = 1 - \frac{2((u + 1) - iv)}{((u+1)+iv)((u+1)-iv)} = \\
		     & = 1 - \frac{2(u + 1 - iv)}{(u+1)^2 + v^2} = 1 - \frac{2u + 2 - 2iv}{u^2 + 2u + v^2 + 1} =     \\
		     & = 1 - \frac{2u+2}{u^2+2u+v^2+1} - i\frac{2v}{u^2+2u+v^2+1} =                                  \\
		     & = \frac{u^2 +v^2 - 1}{u^2+2u+v^2+1} - i \frac{2v}{u^2+2u+v^2+1}
	\end{align*}

	Значит \( \Re(w(z)) = \frac{u^2+v^2-1}{u^2+2u+v^2+1} \), \( \Im(w(z)) = - \frac{2v}{u^2+2u+v^2+1} \)
\end{frame}

\subsection{Точка}
\begin{frame}\frametitle{Точка}
	Изучим, как под действием отображения изменяется точка на плоскости.

	По \href{https://www.geogebra.org/calculator/dncbzxhp}{ссылке} можно перейти на
	демонстрацию Geogebra, на которой находится точка \(A\) в виртуальном пространстве
	и соответствующая ей точка \(A' = w(A)\) в физическом пространстве.

	С включенным режимом трассировки, можно перемещать точку \(A\) и
	изучать, во что переходит соответствующая фигура.
\end{frame}



\subsection{Координатная сетка}
\begin{frame}\frametitle{Координатная сетка}
	Изучим, как под действием отображения изменяется координатная сетка:
	\begin{enumerate}
		\item Построим в виртуальном пространстве множество точек
		      \(v = C\)~--~горизонтальные прямые и \(u = C\)~--~вертикальные прямые
		\item Применим к этим точкам преобразование
		\item Изобразим получившиеся точки в физическом пространстве
	\end{enumerate}

	В приведенных на следующих слайдах графиках константа \(C \in \{-2, -1, -0.5, 0, 0.5, 1, 2\}\)
\end{frame}

\begin{frame}{Координатная сетка (горизонтальные прямые)}
	\begin{figure}
		\centering
		\includegraphics[width=0.9\textwidth]{figures/conformal_grid_horizontal.pdf}
		\caption{Координатная сетка (горизонтальные прямые)}\label{fig:conformal_grid_horizontal}
	\end{figure}
\end{frame}

\begin{frame}{Координатная сетка (вертикальные прямые)}
	\begin{figure}
		\centering
		\includegraphics[width=0.9\textwidth]{figures/conformal_grid_vertical.pdf}
		\caption{Координатная сетка (вертикальные прямые)}\label{fig:conformal_grid_vertical}
	\end{figure}
\end{frame}


\begin{frame}{Координатная сетка}
	\begin{figure}
		\centering
		\includegraphics[width=0.9\textwidth]{figures/conformal_grid_combined.pdf}
		\caption{Координатная сетка}\label{fig:conformal_grid}
	\end{figure}
\end{frame}

\subsection{Геометрические фигуры}
\begin{frame}\frametitle{Влияние отображения на геометрические фигуры}
	Изучим, как меняются геометрические фигуры под действием отображения.

	Как и в прошлом пункте, будем строить фигуры в виртуальном пространстве,
	применять к точкам, лежащим на этих фигурах отображение и строить
	получившиеся точки в физическом пространстве.
\end{frame}

\begin{frame}{Отрезок \(u = v, u \in [-10, 10]\)}
	Видно, что отрезок переходит в часть окружности, незамкнутую
	в окрестности точки \(w = 1\).
	При дальнейшем увеличении отрезка окрестность будет уменьшаться.
	\begin{figure}
		\centering
		\includegraphics[width=\textwidth]{figures/conformal_u_equals_v.pdf}
	\end{figure}
\end{frame}

\begin{frame}{Парабола \(v = u^2, u \in [-4, 4]\)}
	\begin{figure}
		\centering
		\includegraphics[width=\textwidth]{figures/conformal_parabola.pdf}
		% \caption{Парабола \(v=u^2\), \(u \in [-4, 4]\)}
		% \label{fig:conformal_parabola}
	\end{figure}
\end{frame}

\begin{frame}{Окружность \((v-1)^2 + u^2 = 2\)}
	\begin{figure}
		\centering
		\includegraphics[width=\textwidth]{figures/conformal_circle.pdf}
		% \caption{Окружность \((v-1)^2 + u^2 = 2\)}\label{fig:conformal_circle}
	\end{figure}
\end{frame}

\subsection{Обратное преобразование}
\begin{frame}\frametitle{Обратное преобразование}
	Найдем для данного преобразования обратное.
	Для этого выразим \(z(w)\)
	\begin{align*}
		w(z) & = \frac{z-1}{z+1}                      \\
		z(w) & = \frac{1+w}{1-w} = 1 + \frac{2w}{1-w}
	\end{align*}

	Видно, что обратное преобразование конформно за исключением
	простого полюса \(w = 1\).
	Простым нулем обратного преобразования является точка \(w = -1\).

	Полюс \(w = 1\) и объясняет наличие выколотой точки \(w = 1\)
	на предыдущих графиках.
\end{frame}

\subsection{Показатель преломления}
\begin{frame}\frametitle{Профиль показателя преломления}
	Для расчета профиля показателя в физическом пространстве воспользуемся формулой:
	\begin{equation}
		n_z =
		\left|\frac{dw}{dz}\right| n_w =
		\frac{2}{(x+1)^2+y^2}
		\label{eq:retractive_index}
	\end{equation}

\end{frame}

\begin{frame}\frametitle{Профиль показателя преломления}
	\begin{figure}
		\includegraphics[width=0.6\textwidth]{figures/conformal_retractive_index.pdf}
		\caption{Профиль показателя преломления}
	\end{figure}
\end{frame}

\subsection{Вывод по задаче}
\begin{frame}\frametitle{Вывод по задаче}
	\begin{itemize}
		\item Определили особые точки отображения
		\item Изобразили действие отображения на разные кривые
		\item Проанализировали обратное преобразование
		\item Рассчитали профиль показателя преобразования,
		      построили его график
	\end{itemize}
\end{frame}


