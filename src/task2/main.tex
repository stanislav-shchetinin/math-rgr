%Task 
\begin{frame}
	\frametitle{Задание 2. Поток векторного поля}

	Дано тело \(T\), ограниченное следующими поверхностями:

	\begin{equation*}
		y + \sqrt{x^2 + z^2} = 0 \qquad x^2+z^2 = 1 \qquad x^2 + y + z^2 = 2
	\end{equation*}

	\begin{figure}
		\centering
		\includegraphics[width=0.6\textwidth]{figures/2_vec_field_body.pdf}
		\caption{Сечение тела \( T \) координатной плоскостью \(Oyz\) }\label{fig:task2_img}
	\end{figure}
\end{frame}

\begin{frame}
	\frametitle{Задание 2. Поток векторного поля}

	Дано тело \(T\), ограниченное следующими поверхностями:

	\begin{equation*}
		y + \sqrt{x^2 + z^2} = 0 \qquad x^2+z^2 = 1 \qquad x^2 + y + z^2 = 2
	\end{equation*}

	\begin{itemize}
		\item Изобразите тело \(T\) на графике в пространстве.
		\item Вычислите поток поля
		      \begin{equation*}
			      \vec a = (\sin zy^2) \vec i + \sqrt{2} x \vec j + (\sqrt{2+y} -3z) \vec k
		      \end{equation*}
		      через боковую поверхность тела \(T\), образованную вращением дуги \(AFEDC\)
		      вокруг оси \(Oy\), в направлении внешней нормали поверхности тела \(T\).
	\end{itemize}

\end{frame}

\subsection{Тело T на графике в пространстве}
\begin{frame}\frametitle{Тело \(T\) на графике в пространстве }

	\begin{figure}
		\centering
		\includegraphics[width=0.65\textwidth]{figures/2_vec_field_3d_img.pdf}
		\caption{Тело \(\vec T\) в пространстве}\label{fig:vec_field_graph}
	\end{figure}

\end{frame}

\begin{frame}\frametitle{Элементы тела \(T\) на графике в пространстве}

	\begin{figure}[ht]
		\centering
		\begin{minipage}{.48\textwidth}
			\centering
			\includegraphics[width=\linewidth]{figures/2_bottom.pdf}
			\caption{Замкнутое дно тела \(\vec T\) в пространстве}
			\label{fig:vec_field_bottom}
		\end{minipage}\hfill
		\begin{minipage}{.48\textwidth}
			\centering
			\includegraphics[width=\linewidth]{figures/2_rotated_surface.pdf}
			\caption{Поверхность вращения дуги AFEDC вокруг оси \(Oy\)}
			\label{fig:vec_field_rotated_surface}
		\end{minipage}
	\end{figure}

\end{frame}



\subsection{Вычисление потока поля}
\begin{frame}\frametitle{Вычисление потока поля}
	Для нахождения искомого потока, найдем поток через тело вращения
	(Рис. \ref{fig:vec_field_rotated_surface}) и вычтем из него поток
	через конусовидное дно, которое замкнем плоскостью \(y = -1\) (Рис. \ref{fig:vec_field_bottom}):

	\begin{equation*}
		\Phi = \Phi_{\text{вращения}} - \Phi_{\text{дна}}
	\end{equation*}

	Так как в обеих случаях тела замкнуты, то для нахождения
	потока поля через них воспользуемся теоремой \textit{Остроградского -- Гаусса}:
	\begin{equation*}
		\oiint\limits_{\Sigma}\left( \vec {a}, \vec {n} \right) \, d\sigma = \iiint\limits_V \Div \vec {a} \, dx \, dy \, dz
	\end{equation*}

	Найдем дивергенцию:
	\begin{equation*}
		\Div \vec a = \frac{\partial a_x}{\partial x} +  \frac{\partial a_y}{\partial y} +  \frac{\partial a_z}{\partial z} = 0 + 0 - 3 = -3
	\end{equation*}
\end{frame}

\begin{frame}\frametitle{Вычисление потока поля: тело вращения}
	Вычислим поток через тело вращения.
	Перейдем к цилиндрическим координатам:
	\begin{equation*}
		\begin{cases}
			x = r \cdot \cos \theta \\
			y = y                   \\
			z = r \cdot \sin \theta
		\end{cases}
	\end{equation*}

	Расставим пределы интегрирования:
	\begin{align*}
		r \in [0, 1], \
		\theta \in [0, 2\pi], \
		y = 2 - x^2 - z^2 = 2 - r^2
	\end{align*}

	Тогда
	\begin{align*}
		\Phi_{\text{вращения}} & = \oiint\limits_{\Sigma}\left( \vec {a}, \vec {n} \right) d\sigma = \iiint\limits_V -3 dV
		= -3 \int\limits_{0}^{2 \pi} d \theta
		\int\limits_{0}^{1} r~dr
		\int\limits_{0}^{2-r^2} dy =                                                                                       \\
		                       & = -3 \int\limits_{0}^{2 \pi} d \theta
		\int\limits_{0}^{1} (2-r^2)r~dr
		= -3 \cdot 2 \pi \cdot
		\left(1 - \frac{1}{4}\right)
		= - \frac{9}{2}\pi
	\end{align*}
\end{frame}

\begin{frame}\frametitle{Вычисление потока поля: дно тела}

	Расставим пределы интегрирования для конусовидного дна тела:
	\begin{align*}
		r \in [0,1] \
		\theta \in [0, 2\pi] \
		y = -\sqrt{x^2 + z^2} = -\sqrt{r}
	\end{align*}

	Тогда
	\begin{align*}
		\Phi_{\text{дна}}
		 & =  \oiint\limits_{D}\left( \vec {a}, \vec {n} \right) d\sigma = \iiint\limits_D -3 dD
		= -3 \int\limits_{0}^{2 \pi} d \theta
		\int\limits_{0}^{1} r~dr
		\int\limits_{-\sqrt{r}}^{0} dy =                                                         \\
		 & = -3 \int\limits_{0}^{2 \pi} d \theta
		\int\limits_{0}^{1}r^{\frac{3}{2}}~dr
		= -3 \cdot 2 \pi \cdot \frac{2}{5} = - \frac{12}{5} \pi
	\end{align*}

	\begin{equation*}
		\Phi =
		\Phi_{\text{вращения}} - \Phi_{\text{дна}} =
		-\frac{9}{2}\pi - \left(- \frac{12}{5} \pi\right) = - \frac{21}{10} \pi
	\end{equation*}
\end{frame}


\subsection{Вывод по задаче}
\begin{frame}\frametitle{Вывод по задаче}
	\begin{itemize}
		\item Изобразили тело \(T\) на графике в трехмерном пространстве.

		\item Нашли дивергенцию векторного поля \(\Div \vec a\) = -3.

		\item Вычислили поток векторного поля через боковую поверхность тела \(\Phi = -\frac{21}{10}\pi \)
	\end{itemize}

\end{frame}

